\documentclass{beamer}

\usepackage{amsmath,amssymb}
\usepackage{algpseudocode}
\usepackage{algorithm}
\usepackage{tikz}
\usepackage{circuitikz}
\usepackage{graphicx}
\graphicspath{{./prog/}}
\usepackage{svg}
\setcounter{MaxMatrixCols}{12}
%\def\BibTeX{{\rm B\kern-.05em{\sc i\kern-.025em b}\kern-.08em
%    T\kern-.1667em\lower.7ex\hbox{E}\kern-.125emX}}
%\usepackage{bibtex}

\usetheme{Berlin}
\usecolortheme{beaver}
\setbeamertemplate{page number in head/foot}[totalframenumber]

% title information
\title{In-Switch Traffic Distribution Approximation}
\author{Sean Bergen}
\institute{Arizona State University}
\date{2024}


\begin{document}

% title frame
\frame{\titlepage}

% overview/agenda frame
\begin{frame}
  \frametitle{Overview}
  \tableofcontents
\end{frame}

\section{Motivation}
% motivation
\begin{frame}
  \frametitle{Motivation}
  \begin{itemize}
    \item <2-> Categorize traffic rates relative to each other
    \item <3-> Keep operators of algorithm simple to allow for implementation in P4
  \end{itemize}  
\end{frame}

% background info
\section{Background Information}
\begin{frame}
  \frametitle{Background Information}
  \begin{itemize}
    \item Streaming Algorithms
    \item Sketches
    \item Stochastics
  \end{itemize}
\end{frame}


% streaming algs
\subsection{Streaming Algorithms and Sketches}
\begin{frame}
  \frametitle{Streaming Algorithms}
  \begin{itemize}
    \item<2-> Formalized by N. Alon et al\cite{noga}
    \item<3-> Algorithm operates``on-line'' on a stream of data
    \item<4-> Stream usually only examined once
  \end{itemize}
\end{frame}

% sketches
\begin{frame}
  \frametitle{Sketches}
  \begin{itemize}
    \item What if we want to examine stream history as part of the algorithm?
    \item<2-> Probabilistic data structures, highly compressed\cite{sketch}
    \item<3-> Often used alongside streaming algorithms
    \item<4-> Count-Min Sketch\cite{cms}
  \end{itemize}
\end{frame}

% stochastics
\subsection{A (Brief) Introduction to Stochastics}
\begin{frame}
  \frametitle{Random Variables and Stochastic Processes}
  \begin{itemize}
    \item<2-> Random variable $X$ produces events from universe $U$
    \item<3-> Stochastic Process $\rightarrow$ Observing $X$ over time
    \item<4-> A fair coin vs a sequence of coin flips
  \end{itemize}
\end{frame}

% original idea
% \section{The Model}
% \begin{frame}
%   \frametitle{Original Idea for Model}
%   \begin{itemize}
%     \item<2-> Use formalism of STPN
%     \item<3-> Each packet type fires independently according to its own Poisson distribution
%     \end{itemize}
% \end{frame}

% % problem with original idea
% \begin{frame}
%   \frametitle{Problems with STPNs}
%   \begin{itemize}
%     \item<2-> Hard to analyze, requires heavy simulation or heavy computation
%     \item<3-> Time dynamics cause issues
%     \item<4-> STPN doesn't distinguish between values on markers
%   \end{itemize}
% \end{frame}

% current model
\section{Model}
\begin{frame}
  \frametitle{Model}
  \begin{itemize}
    \item<2-> Discrete time, r.v. $X$, switch $S$
    \item<3-> Every timestep $X$ chooses some packet $\sigma$ across all possible ones it could choose $\Omega$ from some distribution
    \item<4-> $X$ forms an input stream of observed packets that $S$ reads from
  \end{itemize}
\end{frame}

% why this model
% \begin{frame}
%   \frametitle{Why this Model?}
%   \begin{itemize}
%     \item<2-> Lose time dynamics, but retain ordering
%     \item<3-> Less complicated, easier to analyze
%     \item<4-> Less operations switch performs
%   \end{itemize}
% \end{frame}

\section{Sketch}
% sketch
\begin{frame}
  \frametitle{Sketch}
  \begin{itemize}
    \item<2-> Packets having closer group numbers means closer rates
    \item<3-> Lower group number means higher rate, inspiration from power laws and Zipf's Law
    \end{itemize}
    \uncover<4->{
  \begin{tabular}{|| c | c | c ||}
    \hline
    Packet Type & Buffer & Group \\
    \hline
    $\sigma_0$ & 0 & 19 \\
    $\sigma_1$ & 0 & 20 \\
    $\sigma_2$ & 1 & 13 \\
    $\sigma_3$ & 2 & 19 \\
    $\sigma_4$ & 1 & 19 \\
    $\sigma_5$ & 0 & 16 \\
    \hline
  \end{tabular}}
\end{frame}

\section{Algorithm}
% slide
\begin{frame}
  \frametitle{Streaming Algorithm}
  \begin{algorithm}[H]
    \tiny
    \caption{$\mathcal{A}$}
    \begin{algorithmic}[1]
      \State $n \gets \mathbb{N}$ \Comment{$n$ is window size}
      \State $m \gets \mathbb{N}, m < n$ \Comment{$m$ is max group value}
      \State Sketch $s$ is empty initially
      \While{Stream not empty}
        \For{$i = 0; i < n; i++$}
          \State Read next packet $p$, $\sigma_p \gets$ ``type''($p$) \Comment{Type could be srcIP, dstIP, ...}
          \If {$\sigma_p \not\in s$}
            \State Add $\langle \sigma_p, 1, m  \rangle$ to $s$ \Comment{$\sigma_p$ acts as a key to $s$}
            \Else
            \State $s[\sigma_p]$ buffer + 1, $s[\sigma_p]$ group $= max(1,$ group - 1)
          \EndIf
          \For{All other $\sigma \neq \sigma_p \in s$}
            \State $s[\sigma]$ buffer - 1
            \If{ $s[\sigma]$ buffer $< 0$}
              \State $s[\sigma]$ buffer = 1, $s[\sigma]$ group = $min(m,$ group + 1)
            \EndIf
          \EndFor
        \EndFor
        \State Send/Save/Process $s$, then clear for next window  
      \EndWhile
    \end{algorithmic}
  \end{algorithm}
\end{frame}

% slide
\section{Experiment Design}
\begin{frame}
  \frametitle{Experiment Design}
  \begin{itemize}
    \item<2-> Generated a stream of 8,000 random packets, 6 different $\sigma$, $n = 40, m = 20$
    \item<3-> Probability distribution over $\sigma$ was $[0.1,0.2,0.4,0.05,0.15,0.1]$
    \item<4-> Saved sketch for each of the 200 windows
  \end{itemize}
\end{frame}

% start of results slide
\section{Results}
\subsection{Theory}
\begin{frame}
  \frametitle{Interesting Theoretical Results}
  \begin{theorem}<2->
    The maximum number of different $\sigma$ that can get to group 1 for a given $n, m$ is $\lfloor \log \left ( \frac{n}{m} \right ) \rfloor + 1, n,m \in \mathbb{N}, n > m$
  \end{theorem}
  \begin{theorem}<3->
    A (loose) lower bound for the number of different $\sigma$ before at least 1 $\sigma$ is guaranteed to be at group $m$ is $\lfloor \frac{n}{2} \rfloor + 1$
  \end{theorem}
  
  \begin{block}{Remark}<4->
    For $m = 2$, there is a tight bound where if the number of different $\sigma$ is $\geq \lfloor \log \left ( \frac{n}{m} \right ) \rfloor + 2$ then there will always be at least $\sigma$ in group $m$
  \end{block}
\end{frame}

\begin{frame}
  \frametitle{More Theory Results}
  \uncover<2->{
  \begin{figure}[!ht]
\centering
\resizebox{0.4\textheight}{!}{%
\begin{circuitikz}
\tikzstyle{every node}=[font=\large]
\draw  (4.25,13.75) circle (0.5cm);
\node [font=\LARGE] at (4.75,13.25) {};
\draw  (5.5,12.5) circle (0.5cm);
\draw  (5.5,11.25) circle (0.5cm);
\draw  (6.75,12.5) circle (0.5cm);
\draw  (6.75,11.25) circle (0.5cm);
\draw  (8,12.5) circle (0.5cm);
\draw  (8,11.25) circle (0.5cm);
\draw  (6.75,10) circle (0.5cm);
\draw  (8,10) circle (0.5cm);
\draw  (8,8.75) circle (0.5cm);
\draw  (8,7.5) circle (0.5cm);
\node [font=\large] at (4.25,13.75) {$\emptyset$};
\node [font=\large] at (5.5,12.5) {3,0};
\node [font=\large] at (6.75,12.5) {2,0};
\node [font=\large] at (8,12.5) {1,0};
\node [font=\large] at (5.5,11.25) {3,1};
\node [font=\large] at (6.75,11.25) {2,1};
\node [font=\large] at (8,11.25) {1,1};
\node [font=\large] at (6.75,10) {2,2};
\node [font=\large] at (8,10) {1,2};
\node [font=\large] at (8,8.75) {1,3};
\node [font=\large] at (8,7.5) {1,4};
\draw  (8,6.25) circle (0.5cm);
\node [font=\large] at (8,6.25) {1,5};
\draw [->, >=Stealth] (5,12.25) -- (5,11.5);
\draw [->, >=Stealth] (5.75,10.75) -- (6.25,10.25);
\draw [->, >=Stealth] (7,9.5) -- (7.5,9);
\draw [->, >=Stealth] (7.5,8.5) -- (7.5,7.75);
\draw [->, >=Stealth] (7.5,7.25) -- (7.5,6.5);
\draw [->, >=Stealth] (8.5,7.75) -- (8.5,8.5);
\draw [->, >=Stealth] (8.5,9) -- (8.5,9.75);
\draw [->, >=Stealth] (8.5,10.25) -- (8.5,11);
\draw [->, >=Stealth] (8.5,11.5) -- (8.5,12.25);
\draw [->, >=Stealth] (7.5,12.25) -- (7,11.75);
\draw [->, >=Stealth] (7,10.75) -- (7.5,10.25);
\draw [->, >=Stealth] (7,12) -- (7.5,11.5);
\draw [->, >=Stealth] (5.75,12) -- (6.25,11.5);
\draw [->, >=Stealth] (6.25,12.25) -- (5.75,11.75);
\draw [->, >=Stealth] (4.5,13.25) -- (5.25,11.75);
\draw [->, >=Stealth] (6,11.25) -- (6,12.25);
\draw [->, >=Stealth] (7.25,10) -- (7.25,11);
\draw [->, >=Stealth] (7.25,11.5) -- (7.25,12.25);
\draw [->, >=Stealth] (7.5,11) -- (7.5,10.25);
\draw [->, >=Stealth] (7.5,12.25) -- (7.5,11.5);
\draw [->, >=Stealth] (7.5,9.75) -- (7.5,9);
\end{circuitikz}
}%
  \end{figure}}
\end{frame}

\begin{frame}
  \frametitle{Using Markov Chain Properties}
  \setlength{\arraycolsep}{2.5pt}

  Consider $p(x) = 0.4$ for a packet in the previous example: \\

  \scriptsize
\uncover<2->{
$
\begin{bmatrix}
  \emptyset & (3,0) & (3,1) & (2,0) & (2,1) & (2,2) & (1,0) & (1,1) & (1,2) & (1,3) & (1,4) & (1,5) \\
0.6 & 0   & 0.4 & 0   & 0   & 0   & 0   & 0   & 0   & 0   & 0   & 0   \\
0   & 0   & 0.6 & 0   & 0.4 & 0   & 0   & 0   & 0   & 0   & 0   & 0   \\
0   & 0.6 & 0   & 0   & 0   & 0.4 & 0   & 0   & 0   & 0   & 0   & 0   \\
0   & 0   & 0.6 & 0   & 0   & 0   & 0   & 0.4 & 0   & 0   & 0   & 0   \\
0   & 0   & 0   & 0.6 & 0   & 0   & 0   & 0   & 0.4 & 0   & 0   & 0   \\
0   & 0   & 0   & 0   & 0.6 & 0   & 0   & 0   & 0   & 0.4 & 0   & 0   \\
0   & 0   & 0   & 0   & 0.6 & 0   & 0   & 0.4 & 0   & 0   & 0   & 0   \\
0   & 0   & 0   & 0   & 0   & 0   & 0.6 & 0   & 0.4 & 0   & 0   & 0   \\
0   & 0   & 0   & 0   & 0   & 0   & 0   & 0.6 & 0   & 0.4 & 0   & 0   \\
0   & 0   & 0   & 0   & 0   & 0   & 0   & 0   & 0.6 & 0   & 0.4 & 0   \\
0   & 0   & 0   & 0   & 0   & 0   & 0   & 0   & 0   & 0.6 & 0   & 0.4 \\
0   & 0   & 0   & 0   & 0   & 0   & 0   & 0   & 0   & 0   & 0   & 1
\end{bmatrix}
$}
\end{frame}

\begin{frame}
  \setlength{\arraycolsep}{2.5pt}
  Probability distribution of the Markov Chain after 5 moves

  \tiny
  \uncover<2->{
    $
    \begin{bmatrix}
  \emptyset & (3,0) & (3,1) & (2,0) & (2,1) & (2,2) & (1,0) & (1,1) & (1,2) & (1,3) & (1,4) & (1,5) \\
0.07776 & 0.10368 & 0.22464 & 0.06912 & 0.13824 & 0.06912 & 0 & 0.1152 & 0.06912 & 0.10752 & 0.01536 & 0.01024 \\
0 & 0 & 0.23328 & 0 & 0.24192 & 0 & 0 & 0.288 & 0 & 0.2112 & 0 & 0.0256 \\
0 & 0.18144 & 0 & 0.10368 & 0 & 0.12096 & 0.1728 & 0 & 0.31104 & 0 & 0.08448 & 0.0256 \\
0 & 0 & 0.23328 & 0 & 0.1728 & 0 & 0 & 0.35712 & 0 & 0.2112 & 0 & 0.0256 \\
0 & 0.07776 & 0 & 0.20736 & 0 & 0.05184 & 0.1728 & 0 & 0.38016 & 0 & 0.08448 & 0.0256 \\
0 & 0 & 0.07776 & 0 & 0.2592 & 0 & 0 & 0.31104 & 0 & 0.24192 & 0 & 0.11008 \\
0 & 0 & 0.1296 & 0 & 0.27648 & 0 & 0 & 0.35712 & 0 & 0.2112 & 0 & 0.0256 \\
0 & 0.07776 & 0 & 0.10368 & 0 & 0.05184 & 0.27648 & 0 & 0.38016 & 0 & 0.08448 & 0.0256 \\
0 & 0 & 0.07776 & 0 & 0.15552 & 0 & 0 & 0.41472 & 0 & 0.24192 & 0 & 0.11008 \\
0 & 0 & 0 & 0.07776 & 0 & 0 & 0.20736 & 0 & 0.36288 & 0 & 0.1152 & 0.2368 \\
0 & 0 & 0 & 0 & 0.07776 & 0 & 0 & 0.20736 & 0 & 0.1728 & 0 & 0.54208 \\
0 & 0 & 0 & 0 & 0 & 0 & 0 & 0 & 0 & 0 & 0 & 1
\end{bmatrix}
$} \\

\uncover<3->{After 5 moves, it is class 3 with probability 0.32832, class 2 with probability 0.27648, class 1 with 0.31744, and was never observed with probability 0.07776} \\

\uncover<4->{We can compute the expected value of the class $\mathbb{E}(x)=$ 1.85536}
\end{frame}


% slide
\subsection{Experimental}
\begin{frame}
  \frametitle{Experimental Results}
  \begin{figure}
    \includesvg[height=0.8\textheight]{./prog/plot}
    \end{figure}
\end{frame}

% slide
\section{Future Work}
\begin{frame}
  \frametitle{Future Work and Open Questions}
  \begin{itemize}
    \item Tighter bound on interval before group $m$ packets are guaranteed
    \item P4 implementation of sketch that chooses $\sigma$ as srcIP (in progress)
    \item Adding time dynamics with minimal increase to computational complexity
    \item Determining sweet spot for $m, n$ in relation to each other
  \end{itemize}
\end{frame}

% references slide here
\section{References}
\begin{frame}
  \tiny
  \frametitle{References}
  \bibliographystyle{unsrt}
  \bibliography{refs}
\end{frame}


% start of appendix here
% slide
\section{Appendix}
\begin{frame}
  \frametitle{Proof for Upper Bound on Group 1 Packets}
  To prove this, we work backwards from the last $\sigma$ to reach group 1 back to the first one
  \begin{itemize}
    \item <2-> $\sigma_1$ needs $m - 1$ packets to get to group 1
    \item <3-> $\sigma_2$ can also only need $m - 1$ packets and end with a buffer of 0
    \item <4-> $\sigma_3$ now needs $2(m - 1)$ packets to end with a buffer of 0 ...
  \end{itemize}
  \uncover<5->{This can be generalized to the following equation:}
  \uncover<5->{\begin{align*}
                 &\sum_{i=0} 2^i \cdot (m - 1) \leq n - (m - 1) \\
                 & = \lfloor \log \left( \frac{n}{m - 1} \right) \rfloor + 1
                \end{align*}}
\end{frame}

\begin{frame}
  \frametitle{Proof for Lower Bound on Group $m$ Packets}
  In order for no packets to be at group $m$, every $\sigma$ must appear more than once. \\

  \uncover<2->{By Pigeonhole principle, if there are $\lfloor n/2 \rfloor + 1$ different $\sigma$ within a window of length $n$, then at least one of them cannot have appeared more than once, so at least one must be in group $m$}

  \uncover<3-> {Can we do better?  Yes, this is an extremely loose bound, my intuition is that the lower bound before $m$ is guaranteed is also logarithmic because of the buffers of each $\sigma$ that we add in a window}
\end{frame}


\end{document}